%!TEX root=demo-paper.tex


\section{Problem Statement}
\label{sec:problem_statement}

Given a database $D$ and a query $Q$, \SeeDB\ considers a number of views 
that can be generated from $Q$ by adding relational operators, as 
we will see below. 
For the purposes of this discussion, we will refer to views and visualizations
interchangeably, since it is straightforward to translate views into visualizations
automatically: for example, there are straightforward rules that dictate
how the view Table~\ref{tab:staplerX} must be transformed to give a visualization 
like Figure~\ref{fig:staplerX}.
Also for this discussion, we will limit the set of potential views to
be those that generate a two column result, much like Table~\ref{tab:staplerX};
however, \SeeDB\ can generate and recommend visualizations corresponding to
multiple (greater than two) column views.
Lastly, for simplicity, 
we ignore {\em binning}: that is, given a view to be visualized,
there are many ways of binning values to give the view. 
For instance, if we have average profits per day, we can bin the days into
months, into weeks, or into years.

We consider a database $D$ with a snowflake schema,
with dimension attributes $A$, measure attributes $M$, and potential
aggregate functions $F$ over the measure attributes.
We limit the class of queries $Q$ posed over $D$ to be
those that select a horizontal fragment of the fact table:
this selection can be done using selection predicates on the fact
table, or on dimension tables via key-foreign-key joins.
Overall, this class of queries allows the analyst to express their interest
in examining facts (i.e., a slice of the dataset)
that satisfy specific conditions.
We denote the result of $Q(D)$ as $D_Q$.

Given such a query, \SeeDB\ considers all views $V_i$
that perform a group-by and some aggregation on $D_Q$.
Since our resulting view must have two columns (though, as mentioned earlier,
\SeeDB\ does generate and recommend views with more than two columns),
$V_i$ can be represented as a triple $(a, m, f)$,
where $m \in M, a \in A, f \in F$, i.e., the view
performs a group-by on $a$ and applies the aggregation function $f$ on 
a measure attribute $m$.
Thus, $V_i (D_Q)$ can be expressed as the following SQL query:
$${\tt SELECT \ } a, f(m) \ \ {\tt FROM} \  D_Q\  {\tt GROUP \ \ BY} \ a$$
We call this {\em target view}. 

As discussed in the previous section, \SeeDB\ evaluates
whether a view $V_i$ is interesting
by computing the deviation between the view applied to the selected data (i.e., $D_Q$) 
and the view applied to the entire database.
The equivalent view on the entire database $V_i (D)$ can be expressed as:
$${\tt SELECT \ } a, f(m) \ \ {\tt FROM} \  D\  {\tt GROUP \ \ BY} \ a$$
We call this the {\em comparison view}.

Note that both these views, the target view and the comparison view
give rise to a result containing two columns: $a$ and $f(m)$.
A two-column table can be converted into a probability
distribution, where we normalize the values of $f(m)$ such that
they sum to $1$ over the various values of $a$.
For our example in Table~\ref{tab:staplerX}, the probability
distribution of $V_i(D_Q)$, denoted as $P[V_i (D_Q)]$, 
is: (Jan: 180.55/538.18, Feb: 145.50/538.18, March: 122.00/538.18,  April: 90.13/538.18)
A similar probability distribution can be derived for $P[V_i (D)]$.

Then, the utility $U$ of a view $V_i$ is defined as the distance
between the probability distribution 
of the target view and the probability distribution of the 
comparison view; formally:
$$ U (V_i) = S ( P[V_i (D_Q)], P[V_i (D)] ) $$
The utility of a view is our measure for whether the target view 
is ``potentially interesting'' as compared to comparison view: 
the higher the value, the more the deviation
from the comparison view, and the more likely the view is to be interesting to the analyst.
Computing distance between probability distributions has
been well studied in the literature, and \SeeDB\ supports a variety of metrics
to compute distance, including:
\squishlist
  \item {\bf Earth Movers Distance (EMD)}~\cite{wikipedia-prob-dist}: Commonly used to
  measure differences between color histograms from images, EMD is a popular metric for comparing
  discrete distributions.
  \item {\bf Euclidean Distance}: The L2 norm or
  Euclidean distance considers the two distributions are points in a high dimensional space and measures the
  distance between them.
  \item {\bf Kullback-Leibler Divergence}~\cite{wikipedia-KL}: K-L divergence
  measures the information lost when one probability distribution is used to approximate
  another.
  \item {\bf Jenson-Shannon Distance}~\cite{wikipedia-JS,entropy-vis}: Based on
  the K-L divergence, this distance measures the similarity between two probability distributions.
\squishend

We are now ready to state the goals of \SeeDB\ formally:
\begin{goal}
Given an analyst-specified query $Q$ on a database $D$, a distance function $S$,
and a positive integer $k$, find $k$ views $V \equiv (a, m, f)$ that
have the largest values of $U(V)$ among all the views that can be represented
using a triple $(a, m, f)$, while minimizing total computation time.
\end{goal}
Thus, \SeeDB\ aims to find the $k$ views (obtained by adding a single aggregate and
group-by operator) that have the largest utility based on the function $U$.

