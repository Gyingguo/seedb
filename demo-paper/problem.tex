%!TEX root=demo-paper.tex


\section{Problem Statement}
\label{sec:problem_statement}

Given a database $D$ and a query $Q$, \SeeDB\ considers a number of views that
can be generated from $Q$ by adding relational operators.
For the purposes of this discussion, we will refer to views and visualizations
interchangeably, since it is straightforward to translate views into
visualizations automatically: for example, there are straightforward rules that
dictate how the view in Table~\ref{tab:staplerX} can be transformed to give a
visualization like Figure~\ref{fig:staplerX}.
Furthermore, we limit the set of candidate views to those
that generate a two-column result via a single-attribute grouping and
aggregation (e.g. Table~\ref{tab:staplerX}). However, \SeeDB\ techniques can
directly be used to recommend visualizations for
multiple column views ($> 2$ columns) that are generated via multi-attribute
grouping and aggregation.

%Lastly, for simplicity, 
%we ignore {\em binning}: that is, given a view to be visualized,
%there are many ways of binning values to give the view. 
%For instance, if we have average profits per day, we can bin the days into
%months, into weeks, or into years.

We consider a database $D$ with a snowflake schema,
with dimension attributes $A$, measure attributes $M$, and potential
aggregate functions $F$ over the measure attributes.
We limit the class of queries $Q$ posed over $D$ to be
those that select one or more rows from the fact table, and denote the results
as $D_Q$. 
%select a horizontal fragment of the fact table:
%this selection can be done using selection predicates on the fact
%table, or on dimension tables via key-foreign-key joins.
%Overall, this class of queries allows the analyst to express their interest
%in examining facts (i.e., a slice of the dataset)
%that satisfy specific conditions.
%We denote the result of $Q(D)$ as $D_Q$.

Given such a query $Q$, \SeeDB\ considers all views $V_i$ that perform a
single-attribute group-by and aggregation on $D_Q$. We represent $V_i$ as a
triple $(a, m, f)$, where $m \in M, a \in A, f \in F$, i.e., the view
performs a group-by on $a$ and applies the aggregation function $f$ on a measure
attribute $m$. We call this the {\em target view}.
%Thus, $V_i (D_Q)$ can be expressed as the following SQL query:
$${\tt SELECT \ } a, f(m) \ \ {\tt FROM} \  D_Q\  {\tt GROUP \ \ BY} \ a$$ 
As discussed in the previous section, \SeeDB\ evaluates
whether a view $V_i$ is interesting
by computing the deviation between the view applied to the selected data (i.e., $D_Q$) 
and the view applied to the entire database.
The equivalent view on the entire database $V_i (D)$ can be expressed as shown
below that we call the {\em comparison view}. 
$${\tt SELECT \ } a, f(m) \ \ {\tt FROM} \  D\  {\tt GROUP \ \ BY} \ a$$
The results of both the above views are tables with two columns, namely $a$ and
$f(m)$. We normalize each result table into a probability distribution, such
that the values of $f(m)$ such that they sum to $1$.
% over the various values of $a$ and the tables can be normalized into
%probability distributions for comparison. To convert each result table 
For our example in Table~\ref{tab:staplerX}, the probability distribution of
$V_i(D_Q)$, denoted as $P[V_i (D_Q)]$, is: (Jan: 180.55/538.18, Feb:
145.50/538.18, March: 122.00/538.18,  April: 90.13/538.18). A similar
probability distribution can be derived for $P[V_i (D)]$.

Given a view $V_i$ and probability distributions for the
target view  ($P[V_i (D_Q)]$) and comparison view ($P[V_i (D)]$), the
{\em utility} of $V_i$ is defined as the distance between these two probability
distributions. Formally, if $S$ is a distance function,
$$ U (V_i) = S ( P[V_i (D_Q)], P[V_i (D)] )$$

The utility of a view is our measure for whether the target view is
``potentially interesting'' as compared to the comparison view:
the higher the utility, the more the deviation
from the comparison view, and the more likely the view is to be interesting.
% Computing distance between probability distributions has
% been well studied in the literature, and \SeeDB\ supports a variety of metrics
% to compute utility, including Earth Movers Distance~\cite{wikipedia-prob-dist}, 
% Euclidean Distance, Kullback-Leibler Divergence (K-L
% divergence)~\cite{wikipedia-KL}, and Jenson-Shannon
% Distance~\cite{wikipedia-JS,entropy-vis}. 
Computing distance between probability distributions has
been well studied, and \SeeDB\ supports a variety of metrics
to compute utility, including Earth Movers Distance, 
Euclidean Distance, Kullback-Leibler Divergence (K-L
divergence), and Jenson-Shannon
Distance. 
In our demonstration, conference attendees can experiment with
different distance metrics and examine how the choice of metric affects view
quality.
% \agp{can make the following into a comma separated list
% of metrics without giving details if you want to save space. I would also get
% rid of the citations, not necessary to cite everything in a demo paper.}
% \squishlist
%   \item {\bf Earth Movers Distance (EMD)}~\cite{wikipedia-prob-dist}: Commonly used to
%   measure differences between color histograms from images, EMD is a popular metric for comparing
%   discrete distributions.
%   \item {\bf Euclidean Distance}: The L2 norm or
%   Euclidean distance considers the two distributions to be points in a high
%   dimensional space and measures the distance between them.
%   \item {\bf Kullback-Leibler Divergence}(K-L divergence)~\cite{wikipedia-KL}:
%   K-L divergence measures the information lost when one probability distribution is used to approximate
%   the other one.
%   \item {\bf Jenson-Shannon Distance}~\cite{wikipedia-JS,entropy-vis}: Based on
%   the K-L divergence, this distance measures the similarity between two probability distributions.
% \squishend
Finally, we note that while other definitions of the comparison views and
utility metrics are possible, we choose the above definitions in this demo.
\begin{problem}
\vspace{-5pt}
Given an analyst-specified query $Q$ on a database $D$, a distance function $S$,
and a positive integer $k$, find $k$ views $V \equiv (a, m, f)$ that
have the largest values of $U(V)$ among all the views that can be represented
using a triple $(a, m, f)$, while minimizing total computation time.
\vspace{-5pt}
\end{problem}
%Thus, \SeeDB\ aims to find the $k$ views (obtained by adding a single aggregate
% and group-by operator) that have the largest utility based on the function $U$.

