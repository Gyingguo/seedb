\section{Demo Walkthrough}
\label{demo-walkthrough}

We propose to demonstrate the functionality of the SeeDB system by means of
analyzing three diverse datasets of practical importance. Users will be able to
explore each of these datasets in real-time by using SeeDB to formulate
queries and find interesting trends in the underlying dataset. Specifically, we
will use the following datasets for demonstration purposes:

\begin{itemize}
  \item {\bf Store Orders dataset}: This is a canonical dataset used in business
  intelligence applications. It consists of information about orders placed in a
  store including products, prices, ship dates, geographical information,
  profits etc. The dataset is well known for its interesting trends and
  richness of various data types. It will show off SeeDB capabilities to
  correctly identify diverse trends in the data and the ability to deal with
  numeric, categorical, time series and geographic data.
  \item {\bf Election Contribution dataset}: This dataset is a great example of
  the kind of data and analysis that must be done by potential users like
  journalists who are not data analysts by trade but often need to find
  interesting trends in datasets. As a result, this use case will help the
  audience guage the intuitiveness of the user interface, ease of use and fast
  response times.
  \item {\bf Medical dataset:} This dataset is an example of a dataset that a
  researcher (here, a clinical researcher) might use over the course of his/her
  work. This data has a schema that is more complex than the the election
  or store one, and is of larger size too. Since this data is usually analyzed
  by experts, in addition to fast provision of insights, the user also cares about
  flexiblity and ``expert'' operations on this data such as statistical
  information, accuracy of visualizations, drill-downs etc.
\end{itemize}

We envision the demonstration workflow to be as follows: The user selects one of
the three dataset from above for analysis. He/she formulates a selection query
using the SeeDB query builder or by using ready-made queries (e.g. selecting
outliers in a column etc.) and submits the query to SeeDB. SeeDB then searches
through the entire space of possible views using techniques and heuristics
described in Section \ref{optimizations} and returns the top-{\it k} views it
considers most interesting. The SeeDB frontend then visualizes the top-{it k} views and
presents them to the user. The user can interact with each of these views and
perform further exploration through operations such as drill-downs (graphically
selecting subsets of data), comparisons of multiple views etc. The user will
also be able to experiment with the effect of choosing different utility metrics
and optimization strategies described in Section \ref{}.
