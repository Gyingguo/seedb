%!TEX root = demo-paper.tex

\section{Demo Walkthrough}
\label{demo-walkthrough}
 
We propose to demonstrate the functionality of \SeeDB\ through hands-on
interaction with a variety of datasets. Our goals are two fold: (1) demonstrate
the utility of a tool like \SeeDB\ in surfacing interesting trends for a query
and (2) demonstrate that we can return high quality results efficiently for
datasets with varying sizes and number of attributes.

\stitle{Demonstrating Utility:} To show the utility of \SeeDB\ in a real-world
scenario, we will provide conference attendees three diverse datasets that they
can explore and interact with. Attendees can pose ad-hoc or pre-selected queries
on various datasets and evaluate the visualizations returned. The
evaluation is based on whether the visualizations surface ``interesting''
aspects of the queried data and whether the right visualizations have been
selected. To aid the evaluation of visualizations, the demo verion of \SeeDB\
will have the option of showing ``bad'' visualizations too, i.e. visualizations
that were predicted to have low utility. The purpose behind providing some
pre-selected queries (and interesting information about their results) is to
allow attendees to confirm that the tool does indeed reproduce known
information about these queries. The attendees will also have the option of
trying various utility metrics as described in Section
\ref{sec:problem_statement}. The demo datasets will include:
 
\begin{denselist}
  \item {\bf Store Orders dataset}~\cite{superstore}: This dataset is
    often used by Tableau~\cite{tableau} as a canonical dataset for
    business intelligence applications. It consists of information
    about orders placed in a store including products, prices, ship
    dates, geographical information, and profits. This dataset
    is well-studied by users learning to use Tableau and has several
    web-pages dedicated to discovering interesting trends hidden in
    it~\cite{website}. Attendees using \SeeDB\ will be able
    to identify very quickly the same insights and trends that Tableau
    users have discovered over many years. This dataset will also
    enable us to demonstrate how \SeeDB can correctly deal with
    numeric, categorical, and geographic data.
  \item {\bf Election Contribution dataset}~\cite{election_data}: This dataset
  is an example of a real-world dataset that is typically analyzed by
    non-expert data analysts, such as journalists or historians. This
    dataset will enable us to demonstrate to the attendees how
    non-experts can quickly arrive at interesting visualizations via
    the intuitive user interface.
  \item {\bf Medical dataset~\cite{mimic}:} This dataset is an example of a
    real-world dataset that a researcher (here, a clinical researcher)
    might use over the course of his/her work. This data has a schema
    that is more complex than the the election or store one, and is of
    larger size too.  
\end{denselist}

\stitle{Demonstrating Speed and Optimizations:} This demonstration
scenario will use an enhanced user interface and synthetic datasets with varying
sizes, attribute numbers and attribute distributions. Attendees can examine the
efficiency of \SeeDB\ by tuning various ``knobs'' such as data size, number of
attributes, and optimizations applied. This scenario will highlight the several
optimizations that must be applied by \SeeDB\ to provide interactive response
times without sacrificing accuracy.

Thus, through our demonstration of \SeeDB\, we seek to illustrate that (a) it is
possible to automate labor-intensive parts of data analysis, (b) aggregate
and grouping-based views are a powerful means of identifying interesting trends
in data, and (c) we can use various optimizations to enable data analysis in real-time for a range
of datasets.

\eat{
We propose to demonstrate the functionality of the SeeDB system by means of
analyzing three diverse datasets of practical importance. Users will be able to
explore each of these datasets in real-time by using SeeDB to formulate
queries and find interesting trends in the underlying dataset. Specifically, we
will use the following datasets for demonstration purposes:

\begin{itemize}
  \item {\bf Store Orders dataset}: This is a canonical dataset used in business
  intelligence applications. It consists of information about orders placed in a
  store including products, prices, ship dates, geographical information,
  profits etc. The dataset is well known for its interesting trends and
  richness of various data types. It will show off SeeDB capabilities to
  correctly identify diverse trends in the data and the ability to deal with
  numeric, categorical, time series and geographic data.
  \item {\bf Election Contribution dataset}: This dataset is a great example of
  the kind of data and analysis that must be done by potential users like
  journalists who are not data analysts by trade but often need to find
  interesting trends in datasets. As a result, this use case will help the
  audience guage the intuitiveness of the user interface, ease of use and fast
  response times.
  \item {\bf Medical dataset:} This dataset is an example of a dataset that a
  researcher (here, a clinical researcher) might use over the course of his/her
  work. This data has a schema that is more complex than the the election
  or store one, and is of larger size too. Since this data is usually analyzed
  by experts, in addition to fast provision of insights, the user also cares about
  flexiblity and ``expert'' operations on this data such as statistical
  information, accuracy of visualizations, drill-downs etc.
\end{itemize}

We envision the demonstration workflow to be as follows: The user selects one of
the three dataset from above for analysis. He/she formulates a selection query
using the SeeDB query builder or by using ready-made queries (e.g. selecting
outliers in a column etc.) and submits the query to SeeDB. SeeDB then searches
through the entire space of possible views using techniques and heuristics
described in Section \ref{optimizations} and returns the top-{\it k} views it
considers most interesting. The SeeDB frontend then visualizes the top-{it k} views and
presents them to the user. The user can interact with each of these views and
perform further exploration through operations such as drill-downs (graphically
selecting subsets of data), comparisons of multiple views etc. The user will
also be able to experiment with the effect of choosing different utility metrics
and optimization strategies described in Section \ref{}.
}
