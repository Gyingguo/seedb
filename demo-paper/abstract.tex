% Data scientists rely on visualizations to interpret the data returned by
% queries, but finding the right visualization remains a manual task that is often
% laborious. We propose a DBMS that partially automates the task of finding the
% right visualizations for a query. In a nutshell, given an input query Q, the new
% DBMS optimizer will explore not only the space of physical plans for Q, but also
% the space of possible visualizations for the results of Q. The output will
% comprise a recommendation of potentially ``interesting'' or ``useful''
% visualizations, where each visualization is coupled with a suitable query
% execution plan. We discuss the technical challenges in building this system and
% outline an agenda for future research.
% 
Data scientists analyzing large amounts of data often rely on visualizations to
identify interesting trends and gain insights. However, picking the right
visualization is a manual and tedious task. Given a query, the analyst would
like to know what makes the results of a query ``interesting'' compared to the
underlying dataset. In this paper, we demonstrate a prototype of a system
\SeeDB\, a system that automatically discovers statistical differences between
the query results and the underlying dataset, and visualizes the differences to
aid data exploration.