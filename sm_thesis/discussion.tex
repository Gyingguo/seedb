\chapter{Discussion}
\label{sec:discussion}
This thesis describes our implementation of SeeDB and its evaluation.
There are several ways to extend SeeDB to be more efficient, more
flexible and more helpful in the data analysis process. We now describe some
directions for future work

\section{Making SeeDB more efficient} Interactive response times for SeeDB are
currently achieved mainly through aggressive sampling and query optimization
strategies. Another way to achieve lower response times is to perform aggressive
pruning of views even before the corresponding view queries are executed by the
DBMS.
This pruning can be performed if we can claim with high probability that certain
views are guaranteed to be less interesting than other views. We can leverage
information about data types, data distributions and correlations in order to
perform this pruning of views.

% We briefly
% discussed some basic work we have done to perform this pruning (Chapter XXX). However, we can prune the space of
% possible views even further using additional properties of the data or some
% pre-computation.

Another approach to making \SeeDB\ more efficient is to choose a backend DBMS
that is particularly suited for the workloads generated by \SeeDB.
The advantage of having \SeeDB\ as a wrapper over
the database is that we can replace backends without changes to the SeeDB code.
In particular, it would be instructive to compare how databases with different
data layouts can speed up \SeeDB\ processing. One may expect that column stores
like Vertica may be more efficient at processing \SeeDB\ workloads since
individual columns would be stored separately. It is however also likely that
optimization strategies would be significantly different depending on
the data layout.
Similarly, a comparison of column stores vs. main memory databases like VoltDB
could also provide interesting insights.
Finally, we can attempt to speed up workloads like \SeeDB\ by implementing
operators inside the database that can leverage shared scans for tables. 

\section{Making \SeeDB\ more flexible} For data analysis tools to be effective,
they must achieve the right balance of automation and interactivity in the analysis
process.
For instance, in \SeeDB, merely providing the analyst the system's pick of ten
most interesting views is insufficient. We must not only provide explanations
for our choice but also allow the user to interrogate our views directly and
further manipulate the data iteratively. 

We can offer the user even more flexibility by providing a diverse set of
distance metrics and allowing the user to specify the distance metric. In the
future, we could also attempt to learn a distance metric based on user's
feedback. Similarly, we can leverage user feedback to learn the type of views
that a user finds interesting and use that model to prune uninteresting views.

\section{Making \SeeDB\ more helpful in data analysis} The model used by \SeeDB\
to measure differences in data is only one of many ways to find differences in
data. One can image applying a host a techniques including statistical
significance testing, classification and clustering for this purpose. In the
future, SeeDB should be augmented to include these additional difference-finding
techniques. It is likely that this would require redefining distance metrics and
completely redesigning optimizations for these operations. However, it would be
possible to use the current aggregate+group-by framework used by \SeeDB\ as a
pruning step for these more advanced techniques.
The addition of this functionally is not a trivial change, but it would make
\SeeDB\ much more useful for data analysts.
