\section{State-of-the-Art Approaches}
\label{sec:related_work}

\stitle{Related Work:}
Over the past few years, the research community has introduced 
a number of interactive data analytics tools such as ShowMe, Polaris, and
Tableau~\cite{DBLP:journals/cacm/StolteTH08, DBLP:journals/tvcg/MackinlayHS07}
as well as tools like Profiler allow analysts to detect anomalies in data.
Unlike \SeeDB, which recommends visualizations automatically, the tools place the
onus on the analyst to specify the visualization to be generated.
Similar visualization specification tools have also been introduced
by the database community, including Fusion Tables~\cite{DBLP:conf/sigmod/GonzalezHJLMSSG10} 
and the Devise~\cite{DBLP:conf/sigmod/LivnyRBCDLMW97} toolkit.
There has been some work on browsing data cubes in OLAP, allowing
analysts to find explanations, get suggestions for next cubes to visit,
or identify generalizations or patterns starting from a single cube~\cite{DBLP:conf/vldb/Sarawagi99, 
DBLP:conf/vldb/SatheS01, DBLP:conf/vldb/Sarawagi00}. 
While we may be able to reuse the metrics from that line of work,
the same techniques will not directly apply to visualizations.

\subsection{Viz papers and tools}
\subsection{Data Cubes}
\subsection{General ML and stats references}
\subsection{Multi-query optimization}