\section{State-of-the-Art Approaches}
\label{sec:related_work}

Over the past few years, there has been a significant
effort from the visualization community to provide interactive tools
for data analysts. In particular, tools such as ShowMe, Polaris, and
Tableau~\cite{DBLP:journals/cacm/StolteTH08,
  DBLP:journals/tvcg/MackinlayHS07} provide a canvas for data analysts
to manipulate and view data, tools such as
Wrangler~\cite{DBLP:conf/chi/KandelPHH11} allow data analysts to
transform and clean data, and tools such as
Profiler~\cite{DBLP:conf/avi/KandelPPHH12} allow users to visualize
simple anomalies in data.  However, unlike \SeeDB, these tools have
little automation; in effect, it is up to the analyst to generate a
two-column view to be visualized. Other related areas of work include OLAP and
database visualization tools. There has been some work on browsing data cubes, allowing
analysts to variously find ``explanations'' for why two cube values were
different, to find which neighboring cubes have similar properties to the cube
under consideration, or get suggestions on what unexplored data cubes should be
looked at next~\cite{DBLP:conf/vldb/Sarawagi99, DBLP:conf/vldb/SatheS01,
DBLP:conf/vldb/Sarawagi00}.

Fusion tables~\cite{DBLP:conf/sigmod/GonzalezHJLMSSG10} allows users to create
visualizations layered on top of web databases; they do not consider the problem
of automatic visualization generation.
Devise~\cite{DBLP:conf/sigmod/LivnyRBCDLMW97} translated user-manipulated
visualizations into database queries.

\subsection{Viz papers and tools}
\subsection{Data Cubes}
\subsection{General ML and stats references}
\subsection{Multi-query optimization}