\section{User Study: is this useful?}
\subsection{Comparison with other tools - Tableau, Spotfire}
\subsection{Lessons from different datasets}

We propose to demonstrate the functionality of \SeeDB\ through hands-on
interaction with a variety of datasets. Our goals are two fold: (1) demonstrate
the utility of a tool like \SeeDB\ in surfacing interesting trends for a query
and (2) demonstrate that we can return high quality results efficiently for
datasets with varying size and number of attributes.\\

\stitle{Demonstrating Utility:} To show the utility of \SeeDB\ in a real-world
scenario, we will provide conference attendees three diverse datasets that they
can explore and interact with. Attendees can pose ad-hoc or pre-selected queries
on any of the following datasets and evaluate the visualizations returned. The
evaluation is based on whether the visualizations surface ``interesting''
aspects of the queried data and whether the right visualizations have been
selected. To aid the evaluation of visualizations, the demo verion of \SeeDB\
will have the option of showing ``bad'' visualizations too, i.e. visualizations
that were predicted to have low utility. The purpose behind providing some
pre-selected queries (and interesting information about their results) is to
allow attendees to confirm that the tool doesn indeed reproduce known
information about these queries. The attendees will also have the option of
trying various utility metrics as described in Section
\ref{sec:problem_statement}. The demo datasets will include:
 
\begin{enumerate}
  \item {\bf Store Orders dataset}~\cite{superstore}: This dataset is
    often used by Tableau~\cite{tableau} as a canonical dataset for
    business intelligence applications. It consists of information
    about orders placed in a store including products, prices, ship
    dates, geographical information, profits, and so on. This dataset
    is well-studied by users learning how to use Tableau, with several
    web-pages dedicated to discovering interesting trends hidden in
    this dataset~\cite{website}. Attendees using \SeeDB\ will be able
    to identify very quickly the same insights and trends that Tableau
    users have discovered over many years. This dataset will also
    enable us to demonstrate how \SeeDB can correctly deal with
    numeric, categorical, and geographic data.
  \item {\bf Election Contribution dataset}~\cite{election_data}: This dataset
  is an example of a real-world dataset that is typically analyzed by
    non-expert data analysts, such as journalists or historians. This
    dataset will enable us to demonstrate to the attendees how
    non-experts can quickly arrive at interesting visualizations via
    the intuitive user interface.
  \item {\bf Medical dataset~\cite{mimic}:} This dataset is an example of a
    real-world dataset that a researcher (here, a clinical researcher)
    might use over the course of his/her work. This data has a schema
    that is more complex than the the election or store one, and is of
    larger size too.  
\end{enumerate}