%!TEX root=document.tex


\section{Discussion}
\label{sec:discussion}
We now describe how \VizRecDB\ can be extended to even more general settings,
and also propose some directions for future research.

\subsection{Generalized Visualizations}\label{sec:discussion:multi-col}
\agp{multi-column viz}

Group-by clauses with multiple dimension attributes: It is
straightforward to extend the \VizRecDB\ techniques  to group-by clauses with multiple attributes.
However, for ease of exposition and visualization, we limit the number of
attributes in the group-by clause to one.

\subsection{Binning}

\subsection{Tables with joins}

\subsection{Access-based Pruning}

\subsection{Finding similar Trends}
The same techniques can be used to find the views where the target and
comparison distribution are most similar. 
This can be useful in cases where the target and comparison datasets are very
different but particular views turn out to be very similar.
This fact is surprising and therefore useful to highlight.

\section{Deviation from Trends}
\agp{Talk about how we may instead want to see the deviation from trends of things that 
are similar rather than the norm. In some sense capture the ``average behavior''
of individual items.}

\subsection{Improving Usability} 
For data analysis tools to be effective,
they must achieve the right balance of automation and interactivity in the analysis
process.
For instance, in \VizRecDB, merely providing the analyst the system's pick of ten
most interesting views is insufficient. 
We must not only provide explanations
for our choice but also allow the user to interrogate our views directly and
further manipulate the data iteratively. 
In the future, we could also attempt to learn a distance metric based on user's
feedback. Similarly, we can leverage user feedback to learn the type of views
that a user finds interesting and use that model to prune uninteresting views.


