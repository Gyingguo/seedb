\section{SeeDB Frontend}
\label{subsec:seedb_frontend}

The \SeeDB\ frontend serves two purposes - to input a query, and to visualize
and analyze the resulting views. We provide the user with three options for
specifying an input query: (a) as raw SQL, (b) through a query builder that can
allow users unfamiliar with SQL to formulate queries through an easy-to-use
form-based interface, (c) through pre-defined query templates, e.g., queries
that select outliers in a particular column. These templates are particularly
useful since users are interested in anomalous data points.

Once a user submits a query to \SeeDB\, the \SeeDB\ engine evaluates various
views and sends the most promising ones to the frontend. The frontend then
determines the best ways to visualize these views (e.g. depending on data types
being represented, number of distinct values etc.) and displays the
visualizations. The user can then examine these diverse views at a glance,
explore specific views in detail and view metadata for each view (e.g. size of
result, sample data, value with maximum change, statistics etc.). The user can
also slice-and-dice views further by selecting particular values of grouping
attributes to explore. The user does this simply by selecting the relevant
attribute values in the view. This automatically modifies the selection query
and displays views for the subset of data selected. The user can of course
revert back to the original views and continue exploring the data.

\subsection{Interactivity. Ship all data to client?}